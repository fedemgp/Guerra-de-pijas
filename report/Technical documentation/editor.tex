El editor fue desarrollado en Qt5 debido a que, al contrario de SDL, posee ciertas facilidades como
entradas de texto, scroll bars, etc.

Utilizando QtCreator se creó una ventana con los controles básicos y una vista dónde se dibuja el nivel.
La vista es un objeto \texttt{EditorView} con su \texttt{EditorScene}, que heredan de \texttt{QGraphicsView}
y \texttt{QGraphicsScene} respectivamente. Estas clases se encargan de manejar la lógica del editor, como
dibujar el "ghost" a modo de cursor, llevar la cuenta de los objectos creados, etc.

\texttt{EditorView} se encarga además de manejar los eventos del usuario (teclado, mouse) de forma tal que
permita el uso adecuado del \texttt{EditorScene}. Este, por su parte, almacena internamente en un \texttt{map}
las vigas y worms para que puedan ser obtenidas mas tarde. También dibuja los fondos para reflejar las elecciones
del usuario, y evita que se inserten objetos que no corresponden, como ser una viga superpuesta a un worm o
un elemento fuera del área definida para el nivel.

Cada elemento del juego (vigas, o worms) son subclases de \texttt{StageElement} (subclase de \texttt{QGraphicsItem}).
Esta interfaz permite que la vista tenga un puntero a un \texttt{StageElement} genérico (el cuál se elige mediante
los controles) que puede utilizar para instanciar elementos en el nivel de forma similar a un prototype.
Estos objetos además implementan su propia serialización, que luego utiliza un objeto \texttt{StageData}
como visitor para recolectar los datos finales. De esta forma crear un nuevo elemento no resulta muy complejo
ya que sólo de debe implementar una nueva subclase de \texttt{StageElement}.

El objeto \texttt{StageData} es el encargado de transformar los datos obtenidos en un YAML que luego la
ventana principal guarda en un archivo seleccionado por el usuario.


\subsection{Conclusiones}

Si bien Qt ofrece bastante funcionalidad "out of the box", no siempre es tan inmediato el uso, ya que
la variedad de parámetros a configurar es muy grande y suelen afectarse entre si. Además, al ser un
framework tan grande, tiene una curva de aprendizaje mas empinada que la de SDL.
